\chapter{Der Client}

Dieses Kapitel beschäftigt sich mit dem Android Client der den Projektnamen AmbientClient trägt. Hierbei wird auf den Aufbau des Client sowie 
den Ablauf einer anzulegenden Umgebung eingegangen.

\section{Anforderungen}
Der Client muss im Stande sein mitzubekommen, dass er sich mit einer neuen Umgebung befindet. Des Weiteren muss er bei einer schon vorhanden Konfiguration 
für die Umgebung im Stande sein, eine Veränderung der Konfigurationen automatisch zu senden ohne das der Benutzer etwas machen muss.
Er muss aufzeigen welche Möglichkeiten die Umgebung bietet. Des Weiteren soll er dem User eine Übersich für seine angelegten Umgebungen bieten und durch kurze Wege 
eine leichte und schlüssige Benutzerführung ermöglichen.   

\section{Aufbau} 
Wie schon im vorherigen Kapitel beschrieben soll der Client die Daten des Benutzer auf seinem Smartphone speichern können. 
Wenn dieser sich dann in seiner Umgebung einfindet sollen die Daten übertragen werden. 
Dabei gibt es natürlich verschieden Aspekte die man betrachten muss. Zum einen, wann der Client die Daten übertragen soll. 
Ansonsten gibt es noch die Frage wann der Client eine neue Umgebung anlegen soll und ob dieses Automatisch passieren soll oder ob der User dies selber machen soll.
\\\\
Es ist natürlich am Benutzerfreundlichsten für den User wenn er so wenig wie möglich selber machen muss, 
ohne das die Anwendung in bevormundet und einfach Einstellungen vornimmt, ohne das der User dies möchte. 
Daher ist es so, dass man beim initialen Start der Anwendung seine Allgemeine Konfiguration eingeben muss. 
Der AmbientClient soll automatisch erkennen wenn er sich in einer neuen Umgebung befindet. 
\\
Nun kann der User für seine Umgebung die beim erstmaligen anlegen für das Hotel oder Auto allgemein gültig ist eine Konfiguration anlegen. 
Besucht der Benutzer zu einem späteren Zeitpunkt das gleiche Hotel, aber bekommt ein besseres Zimmer mit mehr Möglichkeiten. 
So wird beim betreten des Raumes und öffnen der AmbientClients
die Konfiguration übertragen und zugleich er hält der Client die Information das mehr Möglichkeiten zum Konfigurieren in diesem Zimmer vorhanden sind. 
Wenn dies der Fall ist bietet der AmbientClient dem User an eine weitere Umgebung für das Zimmer anzulegen und die zusätzlichen Konfigurationen vorzunehmen.



