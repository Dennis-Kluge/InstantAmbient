\chapter{Ausblick \& Zusammenfassung}
Abschliessend soll dieses Kapitel eine Zusammenfassung über die Erkenntnisse und zukünftigen Möglichkeiten des Forschungsprojektes InstantAmbient geben. 
Nachdem die Anforderungen des Projektes formuliert wurden, konnte ein Überblick zu bestehenden Lösungen gegeben werden. Als ersten großes Thema kam es zur Definition von Konfiguration mit der Erklärung ihrer speziellen Mechanismen und Funktionen. Die Grundlage der Konfigurationen erlaubten es spezielle Lösungsmöglichkeiten und eine Architektur für den Proof of Concept zu entwickeln. Die darauf folgenden Kapitel beschäftigten sich mit dessen Umsetzung.

\section{Gewonnene Erkenntnisse dieser Arbeit}
Insgesamt kann ein positives Resümee über dieses Projekt gezogen werden. Es wurde gezeigt, dass die an InstanAmbient gestellten Anforderung durchaus umsetzbar sind. Mit der Definition von Konfigurationen konnte ein Grundstein für auf diesen Konzept basierende Systeme gelegt werden. Die gewonnenen Erkenntnisse über die Architektur und dem Einsatz der einzelnen Systeme hat gezeigt wie vielfältig das Projekt einsetzbar ist. Die Entscheidung über das Client orientierten Konzept und somit einen klaren Schnitt zwischen Nutzerinteraktion und verarbeitenden Systemen hat sich als richtig erwiesen. Das Systemdesign ist klar strukturiert und in jedem Falle erweiterbar sei es beim Front- oder Backend. Es konnte ein Grundstein für den Ausbau der Systeme gelegt werden.

\section{Probleme während der Projektphase}
Probleme während des Projektes gab es auf mehreren Ebenen, da die Fällung einzelner Entscheidungen Auswirkungen auf das gesamte System haben. Besonders die Entscheidung zwischen dem Client- oder Backend orientierten Ansätzen hatten Auswirkungen auf das gesamte Konzept. Die getroffene Wahl war die richtige. Besonderes Kopfzerbrechen brachte die Gestaltung der UI der Android-App und das Konfigurationsrouting mit sich. Auf jeder Ebene hätten schlechte Lösungen das Projekt zum Fall bringen können. 

\section{Zukünftige Möglichkeiten & Herausforderungen}
Um einen Blick in die Zukunft zu werfen sind mehrere Weiterentwicklungen ausgehend von den hier gewonnenen Erkenntnissen möglich. 
\\\\
Der wohl nächste größere Schritt wäre eine größere Testreihe des Systems in heterogenen Umgebungen. So wären sicherlich unterschiedlichste Szenerien wie etwa die der Fahrzeuge oder Hotelumgebungen denkbar. Als erster Schritt wäre zunächst die Portierung der Beispielimplementierung auf die einzelnen Zielsysteme notwendig. Des Weiteren müsste eine Koppelung mit bestehenden Bus-Systemen etc. stattfinden. Dies wird wohl die herausforderndste Aufgabe sein und zeigen ob die Konzeptionen der Connectors und Actors den Anforderungen dieser Systeme entspricht. Eine Vielzahl von Umgebungen erlauben es weiterhin nicht ohne weiteres in die Infrastruktur einzugreifen. Im Bereich der Automobile ist dieses Phänomen besonders stark. Hierfür sind mit größer Wahrscheinlichkeit Kooperationen mit Herstellern notwendig. Der größte wünschenswerte Fall wäre natürlich der Produktiveinsatz. 
\\\\
Eine weiterer Forschungsaspekt wäre die Untersuchung möglicher Konfigurationsvorhersagen basierend auf dem Wissen um welche Umgebung es sich handelt und einer entsprechenden Datenbasis. Hierbei gilt es die Fragen zu klären woher diese Daten stammen und auf welcher Basis eine Analyse erfolgen kann. In InstantAmbient stecken eine Menge weiterer Möglichkeiten und wir selbst sind gespannt was daraus wird. 