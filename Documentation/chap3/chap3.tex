\chapter{Bekannte Lösungen}
In diesem Kapitel werden die schon auf dem Markt vorhandenen Lösungen aufgezeigt sowie Probleme, welche bei einigen Systemen vorhanden sind.

\section{Gegenwärtiger Stand}
Zum Gegenwärtigen Zeitpunkt sind eine Vielzahl von Möglichkeiten zur Heimautomatisierung vorhanden. 
Manche Systeme haben sich hierbei etabliert und finden häufiger Verwendung als andere. Im Bereich der Automobilindustrie gibt es Ansätze um seine Einstellungen zu Konfigurieren bzw. diese zu speichern.  

\subsection{Allgemeine Situation}
In der heutigen Zeit wird der Drang nach der automatischen Bedienung der Heim- und Arbeitsumgebung immer größer. Die Menschen wollen von jedem Ort ihre Umgebung anpassen können und nach ihren Momentanen Bedürfnissen konfigurieren. Im Bereich der Heimautomatisierung gibt es verschiedene Lösungen die zum 
Einsatz kommen. 

\subsection{Heimautomatisierung}
Dieser Abschnitt wird einen kurzer Überblick über verschieden Systeme zur Heimautomatisierung aufgezeigt. Bei der Heimautomatisierung werden viele Systeme zur Überwachung des Hauses eingesetzt, um gegebenenfalls Einbrecher zu ertappen. Des Weiteren werden sie zur Lichtsteuerung und Energieeinsparung genutzt, sodass beispielsweise die Heizung automatisch reguliert wird. Der Unterschied der meisten Systeme liegt in der Bedienung und dem Aufwand ein solches System zu installieren.  

\subsubsection{FS20-System}
Das FS20-System ist ein Funk-Haussteuerungssystem von der Firma ELV und im Niedrigpreissektor das erfolgreichste System. Es besteht aus einer Vielzahl von Empfänger, Sendern und Sensoren. Man hat die Möglichkeit bequem von seinem PC aus das System zu programmieren und mittels drücken verschiedener Taster die konfigurieren Abläufe durchführen zu lassen. Das System kann über das Telefonnetz sowie über den Rechner gesteuert werden. Dies bringt natürlich auch Nachteile mit sich. So ist das FS-20 nur in einer Richtung ansprechbar. Dies bedeutet, dass ein Sender kein Empfangssignal zurückerhält und so mit nicht weiß ob die Informationen erfolgreich beim Empfänger angekommen sind. 

\subsubsection{HomeMatic}
Das HomeMatic System arbeitet wie das FS20-System mittels Funkübertragung. Es ist möglich eine Vielzahl von Sensoren und Aktoren zu kaufen und diese zu installieren. Die Steuerung erfolgt über eine 
Zentrale, falls man einen erweiterten Funktionsumfang erhalten möchte oder komplexe Automatisierungsabläufe konfiguriert. Es besteht hierbei die Möglichkeit verschiedene Signale zu konfigurieren,  welche bestimmte Einstellungen haben und diese mittels Schaltern zu aktivieren. 
Das System kann sowohl über eine Weboberfläche, als auch über das iPhone, das Telefon und über eine Fernbedienung bedient werden. Die Übertragung erfolgt bidirektional somit wird sichergestellt das es eine Rückmeldung geben kann.
\\
HomeMatic ist nicht wie das FS20-System im Niedrigpreissektor angesiedelt.

\subsubsection{XComfort Funk-System}
Das XComfort Funk-System ist ein sehr hochwertiges System. 
Bei diesem System sind allerdings Kenntnisse im Umgang mit dem Pc nötig, da dieses System mittels einer Konfigurationssoftware und eines 
Programmiergerätes konfiguriert werden muss. Bei Verwendung eines Smartphones oder Tablets zum Zugriff auf das System ist ein Server nötig und weitere Software.

\subsubsection{EIB System}
Das EIB System setzt die Verwendung eines integrierten Bussystems im Gebäude voraus. Dabei lassen sich die Geräte zentral und dezentral bedienen und Überwachen. Es ist der gängige Standard in der heutigen Heimautomatisierung und wird fest integriert. Die Programmierung und Steuerung gestaltet ich schwieriger als bei den anderen Systemen. Des Weiteren ist EIB die kostenintensivste Lösung. 

\subsection{Automobil} 
Im Bereich des Automobils spielt der Autoschlüssel eine immer wichtigere Rolle. Das heute in mehreren Autos verwendete System Keyless Go, ist ein System das dem/der FahrerIn ermöglich sein / ihr Auto zu öffnen und zu starten ohne eine aktive Benutzung des Autoschlüssels. Hersteller die dieses System bei ihren Autos integriert haben, nutzen zum großen Teil, auch den Autoschlüssel als Datenspeicher. Diese haben Informationen über die richtige Sitzposition, die Außenspiegel, die Temperatur und sogar den Lieblingssender des/der FahrerIn gespeichert. Wenn sich der/die FahrerIn also seinem/ihrem Auto nähert und dieses öffnet werden diese von ihm/ihr vorgenommen Einstellungen automatisch eingestellt. 
\\
Bei der Erweiterung des Funktionsumfangs eines Autoschlüssels, liegt der derzeitige Trend im Bereich der Automobilindustrie.
Es wird zum Beispiel mit dem Gedanken gespielt einen Kompass in den Schlüssel einzubauen, der einem Zeigen kann wo sich das Auto befindet und Balkendiagramme die die Entfernung des Autos darstellen. Dies muss natürlich über ein Display erfolgen. 
Die Entwickler haben auch die Vision das FreundInnen ihre Lieblingslieder über den Schlüssel austauschen können und diese dann im Auto abgespielt werden. 
\\\\     
Am Beispiel von BMW kann man diese Erweiterungen auch verfolgen. Sie wollen einen Schlüssel erschaffen der die Kreditkarte, die Flugtickets sowie Haustür- und Hotelzimmerschlüssel ersetzen. 
Neben den Daten für die Kreditkarten enthält dieser Schlüssel auch die Fahrzeugdaten. Dieses System ist natürlich Personengebunden. Es ist ein innovativer Schritt.
Bis dieser allerdings serienreife hat, wird es noch eine Weile dauern. 
\\\\
Zurzeit gibt es einen sehr beliebten Trend, der es verschiedenen Leuten ermöglicht ein Auto zu teilen. Dieser Trend nennt sich Carsharing. Es gibt einige Unternehmen die zusammenarbeiten um diesen Service anzubieten. Eine der wohl größten ist Drive-Now\footnote{https://www.drive-now.com/}. 
\\
Beim Carsharing wird je nach Anbieter ein anderer Weg zum öffnen des Autos gewählt. 
Der wahrscheinlich Fortschrittlichste ist jene, bei der mittels eines RFID-Chips, den man sich auf den Führerschein kleben kann, das Auto geöffnet wird.  
Allerdings hat man nicht die Möglichkeit sich weitere Konfigurationen zu speichern. Dieser Bereich bietet dementsprechend noch viele Möglichkeiten.


\subsection{Aufzeigbare Probleme}
Bei den Systemen in der Heimautomatisierung gibt es zum Teil starke preisliche Differenzen. Daher bieten nicht alle Systeme den gleichen Funktionsumfang bzw. Komfort wie andere. 
Bei Preiswerten Systemen ist man an die ortsanhängige Programmierung der Geräte gebunden, die zum Teile keine bidirektionale Verbindung besitzen. Andere Systemen lassen sich auch von Unterwegs aus, über das Internet, konfigurieren. Diese haben aber einen hohen Preis und können unter umständen nicht jederzeit geändert werden, falls es keine Internetverbindung gibt. Die Daten der Konfiguration werden bei diesen Systemen dementsprechend im Hause gespeichert. 
\\\\
Im Automobilbereich wird wie schon erwähnt der Autoschlüssel zur Speicherung der Personenbezogenen Einstellungen genutzt. Hierbei ist eindeutig Aufzuzeigen, dass die Änderung der Einstellungen nur im Auto erfolgen kann und nicht ortsunabhängig ist. Des Weiteren hat man durch die Wahl des Speicherns auf dem Autoschlüssel nicht die Möglichkeit bei einer Autovermietung oder beim Carsharing dies zu nutzen. Hier könnte man höchsten über das Internet Konfigurationen vornehmen die dann im Auto nach der Authentifizierung und Autorisierung geladen werden.
Dies führt aber zu einer Abhängigkeit von externen Systemen und Speicherlösungen. Bei dem Automobil zeigen sich noch erhebliche Problem der ortsunabhängigen und gesammelten Konfigurationen von den Einstellungen des Autos ab. 
\\\\
Im nächsten Kapitel wird auf die Konfiguration eingegangen die in dieser Forschung die zentrale Rolle spielt.

