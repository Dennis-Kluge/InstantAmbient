\chapter{Bekannte Lösungen}
In diesem Kapitel werden die schon auf dem Mark vorhandenen Lösungen aufgezeigt sowie die Probleme die bei manchen Systemen vorhanden sind.

\section{Gegenwärtiger Stand}
Zum Gegenwärtigen Zeitpunkt ist es so, dass es eine Vielzahl von Möglichkeiten für die Heimautomatisierung gibt. 
Manche Systeme haben sich hierbei etabliert und finden häufiger Verwendung als andere. 

\subsection{Allgemeine Situation}
In der heutigen Zeit wird der Drang nach der automatischen Bedienung der Heim und Arbeitsumgebung immer größer. Die Menschen wollen von jedem Ort 
ihre Umgebung anpassen können und nach ihren Momentanen Bedürfnissen konfigurieren. Im Bereich der Heimautomatisierung gibt es verschiedene Lösungen die zum 
Einsatz kommen. 
\subsection{Bekannte Lösungen}
\subsubsection{FS20-System}
Das FS20-System ist ein Funk-Haussteuerungssystem von der Firma ELV und im Niedrigpreissektor das erfolgreichste System. Es besteht aus einer Vielzahl
von Empfänger, Sendern und Sensoren. Dieses System ist allerdings nur in einer Richtung ansprechbar. Dies bedeutet, dass ein Sender kein Empfangssignal zurückerhält 
und so mit nicht weiß ob die Informationen erfolgreich beim Empfänger angekommen sind. Des Weiteren ist das System nicht routingfähig. Das System kann über das Telefonnetz
sowie über den Rechner gesteuert werden.

\subsubsection{HomeMatic}
Das HomeMatic System arbeitet wie das FS20-System mittels Funkübertragung. Es möglich eine Vielzahl von Sensoren zu kaufen und zu installieren. Die Steuerung erfolgt über eine 
Zantrales, falls man einen erweiterten Fumktionsumfang erhlten möchte. Es besteht die Möglichkeit verschieden Signale zu konfigurieren die bestimmte Einstellungen haben. 
Das System kann sowohl über eine Weboberfläche, als auch über das iPhone bedient werden. HomeMatic ist nicht im niedrigpreissektor angesiedelt.

\subsubsection{XComfort Funk-System}
Das XComfort Funk-System ist ein sehr hochwertiges System welches sich besonders durch die Routingfähigkeit auszeichnet. 
Bei diesem System sind allerdings Kenntnisse im Umgang mit dem Pc nötig, da dieses System mittels einer Konfiguratiosnsoftware und eines 
Programmiergerätes konfiguriert werden muss. Bei Verwendung eines Smartphones oder Tablets zum Zugriff auf das System ist ein Server nötig und eine weitere Software.

\subsubsection{EIB System}
Das EIB System setzt die Verwendugn eines integrieten Bussystems im Gebäude voraus. Dabei lassen sich die Geräte zenztral und dezentral bedienen und Überwachen. 

\subsubsection{Auto}
Im Bereich des Automobils gibt es noch keine bekannten Lösungen. Es gibt beim Carsharing die Möglichkeit ein Auto mittels einer Smartcard sich zu 
Autorisieren und das Auto zu öffnen. Allerdings hat man nicht die Möglichkeit sich weitere Konfigurationen auf dieser Karte zu speichern. 
Es gibt verschieden Smartphone-Apps von diversen Herstellern die den Status des Autos an die App übergeben oder die Musik an den Fahrstil des Autofahrers
anpassen. 
\\ 
Mercedes arbeitet mit dem F125 momentan an einem Auto welches sich mittels gesten und Sprache fernsteuern lassen soll. 
Das Forschungsmodel wird seine Serienreife allerdings erst im Jahr 2025 erhalten.

\subsection{Aufzeigbare Probleme}
Ein aktuelles Problem ist momentan der Speicherort der Daten. Die Systeme sind alle dafür ausgelegt, dass sie als Fernbedienung fungieren.
Dieses Forschungsprojekt hat hingegen das Ziel eine Android Anwendung bereitzustellen die die Konfigurierten Daten auf dem Smartphone bereitstellt und 
diese beim Betreten der Umgebung zu übermitteln. Damit die aktuell von Unterwegs aus auf dem Smartphone konfigurierte Umgebung in die 
Heim- und Arbeitsumgebung übermittelt und konfiguriert werden kann. 
\\
Beim Automobil besteht zu Zeit noch ein weiteres Problem. Es ist hier nicht ohne weiteres Möglich das System der Forschungsarbeit einzubauen. Hierbei muss auf das Bussystem des
Autos zugegriffen werden. Was bestimmte Sicherheitsaspekte Voraussetzt. 


- wo sind aktuelle Probleme?

- unbedingt recherche betreiben 
- was haben wir da, gibt es überhaupt etwas? 
