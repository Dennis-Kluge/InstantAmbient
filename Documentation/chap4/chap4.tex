\chapter{Lösungsstrategie von InstantAmbient}
Nachdem im letzten Kapitel ausführlich der Aufbau und die Verarbeitung von Konfigurationen diskutiert wurde, widmet sich dieses den verarbeitenden Systemen.
Lösungsstrategien bedeutet, dass ein überblick darüber gegeben wird wie im generellen konfigurationsbasierte Dienste aufgebaut sein können.

\section{Anforderungen}
Ein sehr wichtiger Bestandteil den InstantAmbient erfüllen muss ist die Skalierbarkeit. 
Das System muss in einer kleineren Umgebung wie dem Auto als auch in einer großen Umgebung wie ein Großraumbüro oder ein Hotel funktionieren. 
Demzufolge muss das System auch eine sehr große Last aushalten können, wenn es in einem Hotel oder einem Großraumbüro eingesetzt wird. 
Hier ist die Last auf Grund der hohen Anzahl an verschiedene Räume ein ausschlaggebender Punkt. Es muss also möglich sein, 
dass mehrere Leute zur gleichen Zeit ihre Konfiguration senden können und diese dann ohne Probleme vom System verarbeitet werden. 
Des Weiteren muss das System eine verständliche Benutzerführung haben. Dementsprechend muss der Client eine sehr leicht erlernbare Benutzerführung mit kurzen Wegen haben. 
Dies erleichtert dem Benutzer das leichte und schnelle ändern von Konfigurationen.   


\section{Backend lastig}
Es gibt verschiedene Wegen das Projekt InstantAmbient zu realisieren. Zum einen gibt es die Möglichkeit, die gesamten Daten im Backend zu lagern. 
Zum anderen gibt es die Möglichkeit die Gesamten Daten auf dem Client zu Speicher. Dieses wird im nächsten Abschnitt näher erläutert.
\\\\
Entwickelt man das System Backend lästig bedeutet dies, dass die gesamten Daten von InstantAmbient im Backend gelagert werden. 
Dadurch wird auch eine Bentuzerverwaltung benötigt. Ohne diese wäre in diesem Fall keine Zuordnung der Konfigurationsdaten zu dem jeweiligen Benutzer möglich. 
Des Weiteren bedeutet dies, dass eine Benutzerverwaltung angelegt werden muss, damit die Konfigurationsdaten dem Benutzer zugewiesen werden können. 
Damit der Benutzer selber seine Konfiguration vornehmen kann, benötigt das Backend eine Schnittstelle die dem User zur Verfügung gestellt werden kann. 
Dabei gibt es verschiedene Möglichkeiten. Man könnte im Bereich von Hotels, Büros, Autovermietungen ein Terminal bereitstellen bei dem sich der Benutzer anmeldet 
um seine Umgebung auf seine Wünsche anzupassen. Dies führt mit sich, dass das Terminal an einem leicht zu erreichenden Standort steht. Eine andere Möglichkeit ist 
die Verwaltung mittels eines Webclients. Bei der Autovermietung sowie bei Hotels und Büros ist es für den Benutzer nicht unbedingt optimal sich erst noch an 
ein Terminal zu stellen um gewünschte Änderungen vorzunehmen. \\
Ein weitere Aspekt ist die Authentifizierung. Bei der selbstständigen Benutzerverwaltung kann dies sehr einfach mit einem Benutzernamen und einen Passwort erfolgen, 
allerdings ist die bei der Authentifizierung im Auto oder Hotelzimmer zum Laden der Daten nicht von Benutzerfreundlich. Hierbei könnte man natürlich auf Smartphones, 
Smartcards oder RFID-Tags in Form eines runden Schlüsselanhänger zurückgreifen. Dieser würde dann nur zu Authentifizierung dienen, wodurch das System weiß, welches Profil geladen und an die Umgebung geschickt werden muss. Zusätzlich gibt es hierbei noch Möglichkeit das System Zentral oder Dezentral aufzubauen. Das bedeutet wiederum, dass die Daten an die jeweilige Umgebung überführt werden müssen und das Diese einen großen Daten Speicher haben müssen, da eine Vielzahl von Personen die Umgebung nutzen, wie bei einer Vermietung oder einem Hotel. Natürlich können auch nur die Daten in der Umgebung gespeichert werden, von den Personen die sich in der nächsten Zeit in dieser Umgebung befinden. Allerdings haben die Benutzer dann nicht die aktuelle Konfiguration ihrer Umgebung, falls diese am bereit gestellten Terminal noch Änderungen vornehmen.  
Generell wird bei dieser Umsetzung des Systems eine hohe Datenmenge für das Backend anfallen und es wird eine große Speicherkapazität benötigt.

\section{Client lastig}

Im vorherigen Abschnitt wurde angesprochen, dass ein Smartphone zur Authentifiezirung genutzt werden könnte, damit die Daten vom Backend an die Umgebung geschickt werden können. 
\\\\
Ein ganz anderer weg ist, sich das Smartphone oder eine Smartcard als Datenspeicher zur nutze zu machen. Beim Smartphone hat man zusätzlich die Möglichkeit eine App zu entwickeln die den Benutzer ermöglich seine Umgebung zu Konfigurieren und auf dem Gerät zu speichern. Bei einer Smardcard bräuchte man hierfür wiederum ein Lese- und Schreibgerät sowie eine Anwendung um seine Umgebung zu Konfigurieren und auf die Smardcard zu übertragen. \\
Durch die Verwendung so eines externen Datenspeichers müssen die Daten nicht mehr im Backend gespeichert werden, wodurch keine große Speicherkapazität mehr vorhanden sein muss. Es wird auch keine Schnittstelle wie ein Terminal oder ein Webclient mehr benötigt. Da in diesem Fall der Datenspeicher Personengebunden ist, ist auch keine Benutzerverwaltung mehr notwendig. Somit bekommt das Backend eine Fokussierung auf die Verarbeitung der Daten, das weiterleiten der Konfiguration und ansprechen der richtigen Umgebung. Aufgrund der hohen Datenmengen die Entstehen können wenn man sich für ein Clientlstiges System entschließt, ist es sinnvoller das Smartphone als Datenspeicher zu verwenden. 
Das Backend muss in diesem System dem Client mitteilen in welcher Umgebung man sich gerade befindet. Nur so ist es möglich, dass der Client dem Backend die richtige Konfiguration der Umgebung schicken kann. Würde dies nicht passieren, würde das System keinen Nutzen haben.

\section{Proof of concept}

Auf Grund des Mehrwertes den ein Clientlastiges System mit sich bringt, wird InstantAmbient nach diesem Prinzip entwickelt.\\
Dabei wird ein Smartphone mit Android Betriebssystem verwendet. 
Hierfür wird ein Client entwickelt mit dem es möglich ist eine Konfiguration für eine bestimmte Umgebung zu erstellen und diese auch bei belieben zu aktualisieren, egal wo man sich in diesem Moment befindet. Was eine hohe Ortsunabhägigkeit beim Konfigurieren seiner Umgebung zur folge hat. \\   
Das Backend hat dem zur folge die Hauptaufgabe die Konfiguration zu empfangen, aufzubereiten und an die entsprechenden Schnittstellen der Geräte zu senden. \\ 
Da dieses Prinzip eine Generalisierung ermöglicht wurde sich dafür entschieden.  


Im nächsten Kapitel wir die Architektur, Technologie und die einzelnen Komponenten genauer beleuchtet.


- abstrakt halten
- nicht zu sehr in die technische Schiene gehen. 

- wie sehen wir eine mögliche Lösung aus
- Generalisierung ansprechen