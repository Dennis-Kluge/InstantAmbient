\chapter{Die Idee von konfigurationsbasierten Systemen}

Dieses Kapitel widmet sich der Gesamtidee von InstantAmbient und diskutiert die Anwendungsszenarien und den Nutzen des Systems. Des Weiteren soll aufgezeigt werden, welche 
Vorteile konfigurationsbasierte Systeme mit sich bringen. 

\section{Die Vision}
Auf die Vision hinter InstantAmbient wurde bereits in der Einführung kurz eingegangen. Es geht darum ein System so zu konzipieren, mit dessen Hilfe es möglich sein soll 
unterschiedlichste Umgebungen nach den eigenen Bedürfnissen anzupassen. Für ein besseres Verständnis soll kurz der Alltag einer fiktionalen Person gezeichnet werden, welche
bereits ausgiebig konfigurationsbasierte Systeme nutzt. 
\\\\
Wir begleiten einen Tag lang Frau Schröter ihres Zeichens Senior Software Developer bei IBM Deutschland und gerade auf Geschäftsreise in London für eine Schulung zur 
Entwicklung mit Ruby on Rails. Sie nutzt seit geringerer Zeit konfigurationsbasierte Systeme und wendet diese so oft wie möglich an. Nach der Ankunft in Haethrow beschließt 
sie der Mobilität wegen, sich ein Mietwagen bei dem bekannteren Anbieter RentX zu nehmen. Dieser rüstet seit einiger Zeit eine Vielzahl seiner Fahrzeuge mit 
konfigurationsbasierten Systemen aus. Nachdem das Auto angemietet wurde und Frau Schröter ihre Schlüssel erhielt, brauch sie nur noch einzusteigen und mit Hilfe der 
integrierten NFC-Schnittstelle\footnote{Welche genau in Frage kommen, ist Teil der Fragestellung.} des Automobils ihre persönliche Konfiguration mit dem Smartphone und einer 
passenden App überspielen. Da sie bereits mehrere Male mit einem Wagen gleichen Typs gefahren ist, war eine passende Konfiguration vorhanden. So stellt sich vollautomatisch
neben den Außenspiegeln, der Sitz auf die korrekte Position ein und im Autoradio läuft der präferierte Musikstream von Frau Schröter mit entspannenden Jazzklängen. Natürlich
stellt sich die Klimaautomatik auf eine mollige Temperatur von 21°C ein. Perfekte Umstände für eine sichere Fahrt zum Hotel.
\\\\
Standesgemäß wurde das Hotelzimmer bereits Online gebucht und bezahlt. Als Zugangsschlüssel wurde ein Key generiert, welcher in eine bestehende oder neue Konfiguration 
eingebettet werden kann. So ist der unter Umständen lästige Check-In etwa durch Zeitdruck nicht mehr notwendig. Das Zimmer muss nur noch bezogen werden. Da Frau Schröter das
erste mal Kundin bei dieser Hotelkette ist, hat sie bereits während des Fluges eine Konfiguration für das Zimmer angelegt und den Key zugewiesen. Hierfür musste Sie jedoch 
nicht komplett ihre persönlichen Vorlieben nochmals eingeben, sondern konnte bestehende Profile anpassen. Der Schliessvorgang und die Übertragung der Konfiguration kann so 
innerhalb eines kurzen Knopfdruckes geschehen. Frau Schröter ist dank der Londoner Rush-Hour spät dran und hat lediglich Zeit ihre Koffer abzustellen um anschließend gleich
zum Schulungsort weiterzufahren. Auf dem Weg in die Tiefgarage ändert sie die Innentemperatur auf 19°C, da es ihr vorhin im Auto zu warm war. 
\\\\
Am Schulungsort angekommen, erhält jede/r TeilnehmerIn einen Schreibtisch zugewiesen um das gelernte Wissen sofort anwenden zu können. Die Schreibtische sind Prototypen der
Firma InstantDesk und bieten ebenfalls die individuelle Konfiguration. Wie es der Zufall will, hat Frau Schröter bereits ebenfalls seit einigen Wochen einen Tisch gleichen
Typs für Beta-Tests in ihrem Büro. Nachdem die Konfiguration aktiviert wurde, sind Tischhöhe, Verbindungen zum VPN-Netzwerk über den integrierten Router und andere 
Belange automatisch eingestellt. Dies sorgt für einen gelungenen Einstieg in die Schulung. 
\\\\
Nach einem langen Arbeitstag kehrt Frau Schröter zurück in ihr Hotelzimmer, dass bereits bei Ankunft ihren Lieblingsfilm geladen hat. Während sie auf der Couch ein Glas Wein 
genießt, ändert sie die Lichteinstellungen im Schlafzimmer und lässt sich ein Bad ein. 
\\\\
Zugegeben ist die hier gezeigte Vision sehr idealisiert dargestellt. Sie soll aber zeigen in welchen Umfang konfigurationsbasierte Systeme eingesetzt werden können. So sind 
wie gezeigt unterschiedlichste Szenerien vorstellbar, welche sich durchaus miteinander kombinieren lassen. Die kleine Geschichte rund um Frau Schröter soll einen Einblick und 
ein Verständnis dafür liefern, in welcher Art und Weise diese Systeme agieren können.  
Um es vorweg zu nehmen derzeit gibt es kein einziges System, welches diese Aufgaben erfüllen kann. Auch ist es möglich einen ersten Einblick in die mögliche 
Komplexität\footnote{Eine der Zentralfragen.} einer generischen Antwort zu zeigen. 
Teile der Vision werden im Laufe dieser Arbeit extrahiert, und zu einem eigenen Lösungsmöglichkeit zusammengefügt. Am Ende soll ein Proof of Conecpt den Entwurf validieren. 

\section{Warum konfigurationsbasierte Systeme?}
Im letzten Abschnitt wurden bereits unterschiedlichste Anwendungsszenarien aufgezeigt. Natürlich lässt sich provokant die Frage stellen warum überhaupt konfigurationsbasierte
Systeme angebracht sind. Um nicht zu viel dem 3. Kapitel vorwegzunehmen sind derzeit nur wenige Umgebungen konfigurierbar, was an unterschiedlichen Gründen 
liegt\footnote{Mehr dazu im nächsten Kapitel.}.
\\\\
Eine Untermenge der Vorteile lassen sich dennoch bereits aus der Vision herauslesen neben der Befriedigung der eigenen Bedürfnisse, haben besonders die Anbieter solcher 
Umgebungen Vorteile. So ist dem Autovermieter RentX daran gelegen, dass seine Kunden so sicher wie möglich das Auto bewegen um das Risiko auf Schäden zu minimieren. Die 
FahrerInnen finden stets ihre Konfiguration vor, welche in den meisten Fällen durch andere Fahrten erprobt ist.
\\\\
Hotelketten können einen besonderen Service anbieten. Neben der individuellen Gestaltung der Zimmer, ist es möglich Vorgänge wie Check-Ins zu vermeiden und die Schlüssel 
bereits vorab zu übertragen. Bei weiterer Betrachtung ist es sogar möglich zu sagen, dass die Bedienung von Multimedia-Geräten nichts anderes als eine Änderung der aktuellen
Konfiguration ist. Die Wege welche hier bestritten werden können, sind durchaus vielseitig. 
\\\\
Zu guter Letzt ist es möglich auch exotischere Beispiele zu finden wie etwa die Konfiguration von Schreibtischen und den damit verbundenen Netzwerken zu ermöglichen. 
Die Möglichkeiten können bis zum Intelligenten Energiemanagement und darüber hinaus gehen.
Warum ein Forschungsbemühen in diese Richtung angebracht ist, liegt auf der Hand. So können eine Vielzahl an Parteien einen praktischen Nutzen durch die Anwendung von 
konfigurationsbasierten Systemen ziehen.

\section{Gewählte Ansätze von InstantAmbient}
Innerhalb der Projektarbeit ist es alleine aus zeitlichen Gründen die Behandlung aller einzelnen Teilaspekte nicht durchführbar. So wird sich innerhalb der nächsten Kapitel 
speziell auf die Teilbereiche der Automobile und Hotelzimmer gestützt, jedoch immer mit dem Anspruch eine generischen Ansatz zu produzieren, welcher sich auf neue 
Anwendungsgebiete ausweiten lässt. Im besonderen Hauptaugenmerk liegt der Aufbau und die Gestaltung von Konfigurationsdateien, wobei eine Vielzahl von unterschiedlichen 
Mechanismen berücksichtigt werden, sowie die Konzipierung der App und Backenddienste\footnote{Damit sind all die Dienste gemeint, welche die Konfigurationen entgegen nehmen 
und weiterverarbeiten.}. Im Fokus stehen die Probleme der Informations- und Datenverarbeitung, weiterführende Überlegungen wie etwa zu Geschäftsmodellen werden nicht 
angestellt. 
\\\\
Das nächste Kapitel widmet sich zunächst der Betrachtung bestehender Lösungen um anschließend vollends die Lösungsstrategie von InstantAbient aufzuzeigen. Hierfür wird 
zunächst die Gestaltung von Konfigurationen diskutiert um anschließend die datenverarbeitenden Mechanismen zu konstruieren. 

