\chapter{Einleitung}

In Zeiten der digitalen Revolution dringen Gadgets jeglicher Art in die Wohnzimmer, Hosentaschen oder gar Körper von ihren UserInnen vor. So werden moderne Gerätschaften 
nicht mehr als ein Beiwerk zur Unterstützung des Alltags gesehen, sondern fast als ein weiteres Sinnesorgan zur Rezeption der Umwelt interpretiert [Verweis zum letzten CR]. 
Jedoch ist bei all dem Drang der Miniaturisierung und des Vordringens in die kleinsten Lebensbereiche ein interessanter Trend zu beobachten. 
Zum größten Teil die eigenen Lebensräume von diesem Fortschritt nur mäßig betroffen. In den heutigen Wohnzimmern finden sich zwar Laptops, Flachbildfernseher und viele andere 
Tools wieder, jedoch ist die eigentliche Infrastruktur bestehend aus Thermostaten, Jalousien usw. wenig betroffen. Die digitale Ansteuerung gestaltet sich als schwer und
kostenintensiv\footnote{Besonders bei der Nachrüstung bestehender Gebäude.}, weiterführende  Konzepte in deren Umgang sind rar gesät. Besonders bei der Annahme, dass alle 
wichtigen Komponenten eines Haushalts, Büros oder Hotelzimmer digital ansteuerbar sind, ergeben sich interessante Ideen. So sollte die Anpassung der Umgebung nach den eigenen
 Bedürfnissen, abspeicherbar in unterschiedlichen Szenarien eiene logische Konsequenz sein. 
Wenn der Gedanke der Konfiguration von Lebensumgebungen mit bereits bestehender und tiefgreifender Infrastruktur weitergeführt wird, ist es leicht auf weitere und 
unterschiedliche Anwendungsszenarien zu blicken. 
So sind beispielsweise Autos ideelle Umgebungen um diese den Wünschen des Fahrer entsprechend zu konfigurieren. Neben Temperatur und Radiosendern können Verkehrsrelevante 
Informationen wie etwa die Sitz- und Spiegelpositionen dem Auto bekannt gemacht werden. Als einer der elementaren Frage bleibt offen, in welcher Form die Konfigurationen 
erstellt und verwaltet werden können? Die Antwort liegt buchstäblich in der Hosentasche, so bieten Smartphones die idealen Bedingungen, dank ihrer allgegenwwärtigen Präsenz 
und ihres User-Interface.
\\\\
Genau dem eben angerissenen Szenario, möchte sich diese Forschungsarbeit widmen. Derzeit gibt es genau für diese Anwendungsfälle wenig konkrete Lösungen, besonders im 
Umgang und der Übertragung von Konfigurationen. Dabei soll auf eine Vielzahl von unterschiedlichen Lebensumgebungen sei es Autos oder Hotelzimmer eingegangen werden, um 
eine möglichst generisches und skalierbares Prinzip zu konstruieren. 
\\\\
Auf den folgenden Kapiteln wird ausgehend von einer genaueren Skizzierung des aktuellen Stands und der Projektidee, verschiedene Lösungen diskutiert. Neben den Übertragungs- 
und Verteilungswegen, wird ein genauer Augenmerk auf die Gestaltung der Konfiguarationen gelegt. Anschließend sollen die entwickelten Strategien in ein Proof of Concept 
überführt werden. Abschliessend wird ein Fazit über die Ergebnisse dieses Forschungsprojekts gegeben. 
\\\\
Diese Arbeit möchte bewusst eine Vision zeichnen, wie der Umgang mit den vorgestellten Technologien in Zukunft aussehen könnte. In dem hier vorgestellten Umfang sind wenige
Ansätze bekannt, daher wurde sich bewusst dafür entschieden einen möglichst generischen Weg zu gehen. Damit auf den Folgenden Seiten nicht stets von \glqq Dem Projekt\grqq 
gesprochen werden muss, wurde sich für den Arbeitstitel \emph{InstantAmbient}\footnote{Basierend darauf, dass die eigene Konfiguration instantan das vorherrschende Ambiente 
ändert. Die Schreibweise im CamelCase orientiert sich an den technischen Aspekt des Projektes.} entschieden.

