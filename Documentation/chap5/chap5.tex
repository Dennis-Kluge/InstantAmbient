\chapter{Architektur & Technologien}


\section{Gesamtarchtitektur}

\section{Aufbau Client}
\section{Aufbau Backend}


\section{Technologien}

\subsection{NFC}
NFC
- grosser Trend 
- für dieses Szenario geeignet, aber es können dienge wie realtime Configuration nicht bequem abgehandelt werden 
- große Konfigurationsdateien: zu geringe Datenrate + NDEF-Format ungeeignet 
- Probleme bei den Implementierungen bei großen Messages => libnfc
- Libraries auf dem aktuellen Stand

Bluetooth
- beliebte Technologie 
- haben nahezu alle Smartphones
- geeignet für diese Art von Datenübertragung 
- libraries nicht gepflegt aber verwendbar
- realtime changes
- für das proof of concept einfacher zu verwenden
- noch immer im Sinne von "NFC"

- Entscheidung für Bluetooth
- das Gesamtkonzept sieht vor den Connector austauschabr zu machen daher kann eine Implementierung von NFC in Zukfunft vorgenommen werden


2 Große Bereiche Client und "Backend" (was eine Vielzahl an Komponenten beschreibt)

- allgemeine Übersicht zu den Komponenten und noch mal Workflow wiederholen
- Namensgebung bzgl Ambient*
- Übersicht siehe Omnigraffle Sketch 

Anforderungen:
- sind unterschiedlich und teilweise auch gegensätzlich wg. den Komponenten 
- leicht verständlich und handhabar auf der User-Seite und dem Client 
- skalierbar, rock solid im Backend

Kommunikation:
Bluetooth

ansonsten messaging Systeme -> das muss noch aufgearbeitet werden wahrscheinlich AMQP

Client

Technologie
- Android, Java, eventuell CouchDB
- geht nicht einfach


- generell Ruby geeignet fürs Prototyping, für Produktivbetrieb eventuell nicht geeignet
AmbientConnector

Technologie
- JRuby 
- BlueCove 
- Blueooth 

AmbientBrain 
- Eventmachine

Technologie
- Ruby 
- Eventmachine
- AMQP

AmbientActor

Technologie
- Arduino n paar Bauteile fürs Konzept
- Ein Ruby Client ebenfalls mit AMPQ

