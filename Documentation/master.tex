\documentclass[11pt, a4paper]{book}

% Test mit dem a5 format
%\usepackage[a5paper]{geometry}

% Silbentrennung
\usepackage[ngerman]{babel}

% Umlaute
\usepackage[utf8]{inputenc}

% Pfeile etc.
\usepackage{amsmath}

% Absatz Einrückung 
\setlength{\parindent}{0pt} 

% Silbentrennung
\usepackage[ngerman]{babel}
\usepackage[T1]{fontenc}

% Tabellen
\usepackage{multirow}

%Source Code einfügen
\usepackage{listings}
\usepackage{float}
\usepackage{color}
\usepackage{textcomp}
\lstset{numbers=left, numberstyle=\tiny, numbersep=5pt, tabsize=2,showstringspaces=false, basicstyle=\footnotesize }


% Bibtex
%\usepackage{natbib}
\usepackage{amsrefs}

% Verweise in PDF		
\usepackage{hyperref} 
\hypersetup{
	colorlinks,
	citecolor=black,
	filecolor=black,
	linkcolor=black,
	urlcolor=black,
	pdftitle={NFC basierte Konfiguration von Lebens- und Arbeitsumgebung},
	pdfauthor={Swen Hutop, Dennis Kluge},
	pdfsubject={Forschungsprojekt},
	pdfproducer={our mac}, % producer of the document
	pdfkeywords={Auto} {Heim} {Automatisierung}, % list of keywords
}

% Glossar
%\usepackage{glossaries}
%\makeglossaries

% für Bilder
\usepackage{graphicx}

% index
%\usepackage{makeidx}
%\makeindex

\begin{document}

\title{NFC basierte Konfiguration von Lebens- und Arbeitsumgebung}
\author{Swen Hutop, Dennis Kluge}

%\pagenumbering{Roman}

% Deckblatte
\thispagestyle{empty}
\begin{center}
\Large{Hochschule für Technik und Wirtschaft Berlin}\\
\end{center}


\begin{center}
\Large{Fachbereich Wirtschaftswissenschaften II}
\end{center}
\begin{verbatim}





\end{verbatim}
\begin{center}
\textbf{\LARGE{Forschungsprojekt 1}}
\end{center}
\begin{verbatim}


\end{verbatim}
\begin{center}
\textbf{im Studiengang Angewandte Informatik}
\end{center}
\begin{verbatim}
















\end{verbatim}

\begin{flushleft}
\begin{tabular}{lll}
\textbf{Thema:} & & NFC basierte Konfiguration von Lebens- und Arbeitsumgebung\\
& & \\
& & \\
& & \\
\textbf{eingereicht von:} & & Swen Hutop  \flq{}swen.hutop@student.htw-berlin.de\frq{}\\
\textbf{eingereicht von:} & & Dennis Kluge  \flq{}dennis.kluge@student.htw-berlin.de\frq{}\\
& & \\
\textbf{eingereicht am:} & & 30. Maärz 2012\\
& & \\
& & \\
\textbf{Betreuer:} & & Herr Prof. Dr. Jürgen Sieck \\
\end{tabular}
\end{flushleft}



% Inhaltsverzeichnis
\tableofcontents


\newpage
%\clearpage\null\vfill


% Abstract
\input{abstract/abstract}

% Alle notwendigen includes

% glossary -  muss als erstes damit die Einträge bekannt sind
%\vfill\null
%\clearpage
\newpage


\chapter{Einleitung}

- warum dieses Thema
- Darstellung des aktuellen Zustands in konfigurationsbasierten Umgebungen 
- aufzeigen der Problematik 
- was möchte diese Arbeit erreichen 
\chapter{Die Idee von konfigurationsbasierten Systemen}

Dieses Kapitel widmet sich der Gesamtidee von InstantAmbient und diskutiert die Anwendungsszenarien und den Nutzen des Systems. Des Weiteren soll aufgezeigt werden, welche 
Vorteile konfigurationsbasierte Systeme mit sich bringen. 

\section{Die Vision}
Auf die Vision hinter InstantAmbient wurde bereits in der Einführung kurz eingegangen. Es geht darum ein System so zu konzipieren, mit dessen Hilfe es möglich sein soll 
unterschiedlichste Umgebungen nach den eigenen Bedürfnissen anzupassen. Für ein besseres Verständnis soll kurz der Alltag einer fiktionalen Person gezeichnet werden, welche
bereits ausgiebig konfigurationsbasierte Systeme nutzt. 
\\\\
Wir begleiten einen Tag lang Frau Schröter ihres Zeichens Senior Software Developer bei IBM Deutschland und gerade auf Geschäftsreise in London für eine Schulung zur 
Entwicklung mit Ruby on Rails. Sie nutzt seit geringerer Zeit konfigurationsbasierte Systeme und wendet diese so oft wie möglich an. Nach der Ankunft in Haethrow beschließt 
sie der Mobilität wegen sich ein Mietwagen bei dem bekannteren Anbieter RentX zu nehmen. Dieser rüstet seit einiger Zeit eine Vielzahl seiner Fahrzeuge mit 
konfigurationsbasierten Systemen aus. Nachdem das Auto angemietet wurde und Frau Schröter ihre Schlüssel erhielt, brauch sie nur noch einzusteigen und mit Hilfe der 
integrierten NFC-Schnittstelle\footnote{Welche genau in Frage kommen, ist Teil der Fragestellung.} des Automobils ihre persönliche Konfiguration mit dem Smartphone und einer 
passenden App überspielen. Da sie bereits mehrere Male mit einem Wagen gleichen Typs gefahren ist, war eine passende Konfiguration vorhanden. So stellt sich vollautomatisch
neben den Außenspiegeln der Sitz auf die korrekte Position ein und im Autoradio läuft der präferierte Musikstream von Frau Schröter mit entspannenden Jazzklängen. Natürlich
stellt sich die Klimaautomatik auf eine mollige Temperatur von 21°C ein. Perfekte Umstände für eine sichere Fahrt zum Hotel.
\\\\
Standesgemäß wurde das Hotelzimmer bereits Online gebucht und bezahlt. Als Zugangsschlüssel wurde ein Key generiert, welcher in eine bestehende oder neue Konfiguration 
eingebettet werden kann. So ist der unter Umständen lästige Check-In etwa durch Zeitdruck nicht mehr notwendig. Das Zimmer muss nur noch bezogen werden. Da Frau Schröter das
erste mal Kundin bei dieser Hotelkette ist, hat sie bereits während des Fluges eine Konfiguration für das Zimmer angelegt und den Key zugewiesen. Hierfür musste Sie jedoch 
nicht komplett ihre persönlichen Vorlieben nochmals eingeben, sondern konnte bestehende Profile anpassen. Der Schliessvorgang und die Übertragung der Konfiguration kann so 
innerhalb eines kurzen Knopfdruckes geschehen. Frau Schröter ist dank der Londoner Rush-Hour spät dran und hat lediglich Zeit ihre Koffer abzustellen um anschließend gleich
zum Schulungsort weiterzufahren. Auf dem Weg in die Tiefgarage ändert sie die Innentemperatur auf 19°C, da es ihr vorhin im Auto zu warm war. 
\\\\
Am Schulungsort angekommen, erhält jede/r TeilnehmerIn einen Schreibtisch zugewiesen um das gelernte Wissen sofort anwenden zu können. Die Schreibtische sind Prototypen der
Firma InstantDesk und bieten ebenfalls die individuelle Konfiguration. Wie es der Zufall will, hat Frau Schröter bereits ebenfalls seit einigen Wochen einen Tisch gleichen
Typs für Beta-Tests in ihrem Büro. Nachdem die Konfiguration aktiviert wurde, sind Tischhöhe, Verbindungen zum VPN-Netzwerk über den integrierten Router und andere 
Belange automatisch eingestellt. Dies sorgt für einen gelungenen Einstieg in die Schulung. 
\\\\
Nach einem langen Arbeitstag kehrt Frau Schröter zurück in ihr Hotelzimmer, dass bereits bei Ankunft ihren Lieblingsfilm geladen hat. Während sie auf der Couch ein Glas Wein 
genießt, ändert sie die Lichteinstellungen im Schlafzimmer und lässt sich ein Bad ein. 
\\\\
Zugegeben ist die hier gezeigte Vision sehr idealisiert dargestellt. Sie soll aber zeigen in welchen Umfang konfigurationsbasierte Systeme eingesetzt werden können. So sind 
wie gezeigt unterschiedlichste Szenerien vorstellbar, welche sich durchaus miteinander kombinieren lassen. Die kleine Geschichte rund um Frau Schröter soll einen Einblick und 
ein Verständnis dafür liefern, in welcher Art und Weise diese Systeme agieren können.  
Um es vorweg zu nehmen derzeit gibt es kein einziges System, welches diese Aufgaben erfüllen kann. Auch ist es möglich einen ersten Einblick in die mögliche 
Komplexität\footnote{Eine der Zentralfragen.} einer generischen Antwort zu zeigen. 
Teile der Vision werden im Laufe dieser Arbeit extrahiert, und zu einem eigenen Lösungsmöglichkeit zusammengefügt. Am Ende soll ein Proof of Conecpt den Entwurf validieren. 

\section{Warum konfigurationsbasierte Systeme?}
Im letzten Abschnitt wurden bereits unterschiedlichste Anwendungsszenarien aufgezeigt. Natürlich lässt sich provokant die Frage stellen warum überhaupt konfigurationsbasierte
Systeme angebracht sind. Um nicht zu viel dem 3. Kapitel vorwegzunehmen sind derzeit nur wenige Umgebungen konfigurierbar, was an unterschiedlichen Gründen 
liegt\footnote{Mehr dazu im nächsten Kapitel.}.
\\\\
Eine Untermenge der Vorteile lassen sich dennoch bereits aus der Vision herauslesen neben der Befriedigung der eigenen Bedürfnisse, haben besonders die Anbieter solcher 
Umgebungen Vorteile. So ist dem Autovermieter RentX daran gelegen, dass seine Kunden so sicher wie möglich das Auto bewegen um das Risiko auf Schäden zu minimieren. Die 
FahrerInnen finden stets ihre Konfiguration vor, welche in den meisten Fällen durch andere Fahrten erprobt ist.
\\\\
Hotelketten können einen besonderen Service anbieten. Neben der individuellen Gestaltung der Zimmer, ist es möglich Vorgänge wie Check-Ins zu vermeiden und die Schlüssel 
bereits vorab zu übertragen. Bei weiterer Betrachtung ist es sogar möglich zu sagen, dass die Bedienung von Multimedia-Geräten nichts anderes als eine Änderung der aktuellen
Konfiguration ist. Die Wege welche hier bestritten werden können, sind durchaus vielseitig. 
\\\\
Zu guter Letzt ist es möglich auch exotischere Beispiele zu finden wie etwa die Konfiguration von Schreibtischen und den damit verbundenen Netzwerken zu ermöglichen. 
Die Möglichkeiten können bis zum Intelligenten Energiemanagement und darüber hinaus gehen.
Warum ein Forschungsbemühen in diese Richtung angebracht ist, liegt auf der Hand. So können eine Vielzahl an Parteien einen praktischen Nutzen durch die Anwendung von 
konfigurationsbasierten Systemen ziehen.

\section{Gewählte Ansätze von InstantAmbient}
Innerhalb der Projektarbeit ist es alleine aus zeitlichen Gründen die Behandlung aller einzelnen Teilaspekte nicht durchführbar. So wird sich innerhalb der nächsten Kapitel 
speziell auf die Teilbereiche der Automobile und Hotelzimmer gestützt, jedoch immer mit dem Anspruch eine generischen Ansatz zu produzieren, welcher sich auf neue 
Anwendungsgebiete ausweiten lässt. Im besonderen Hauptaugenmerk liegt der Aufbau und die Gestaltung von Konfigurationsdateien, wobei eine Vielzahl von unterschiedlichen 
Mechanismen berücksichtigt werden, sowie die Konzipierung der App und Backenddienste\footnote{Damit sind all die Dienste gemeint, welche die Konfigurationen entgegen nehmen 
und weiterverarbeiten.}. Im Fokus stehen die Probleme der Informations- und Datenverarbeitung, weiterführende Überlegungen wie etwa zu Geschäftsmodellen werden nicht 
angestellt. 
\\\\
Das nächste Kapitel widmet sich zunächst der Betrachtung bestehender Lösungen um anschließend vollendsw die Lösungsstrategie von InstantAbient aufzuzeigen. Hierfür wird 
zunächst die Gestaltung von Konfigurationen diskutiert um anschließend die datenverarbeitenden Mechanismen zu konstruieren. 


\chapter{Bereits bestehende Lösungen}
\chapter{Lösungsansatz}

\section{Anforderungen}


\section{Backend lastig}

\section{Client lastig}

\section{Proof of concept}


- abstrakt halten
- nicht zu sehr in die technische Schiene gehen. 

- wie sehen wir eine mögliche Lösung aus
- Generalisierung ansprechen
\chapter{Architektur & Technologien}


\section{Gesamtarchtitektur}

\section{Aufbau Client}
\section{Aufbau Backend}


\section{Technologien}
2 Große Bereiche Client und "Backend" (was eine Vielzahl an Komponenten beschreibt)

- allgemeine Übersicht zu den Komponenten und noch mal Workflow wiederholen
- Namensgebung bzgl Ambient*
- Übersicht siehe Omnigraffle Sketch 

Anforderungen:
- sind unterschiedlich und teilweise auch gegensätzlich wg. den Komponenten 
- leicht verständlich und handhabar auf der User-Seite und dem Client 
- skalierbar, rock solid im Backend

Kommunikation:

Client

Technologie
- Android, Java, eventuell CouchDB

AmbientConnector

Technologie
- JRuby 
- BlueCove 
- Blueooth 

AmbientBrain 

Technologie
- Ruby 
- Eventmachine
- AMQP

AmbientActor

Technologie
- Arduino n paar Bauteile 
- Ein Ruby Client ebenfalls mit AMPQ


\chapter{Konfigurationen}

Dieses Kapitel beschäftigt sich mit dem generellen Aufbau von Konfigurationen welche zwischen dem Smartphone und einem Connector ausgetauscht wird. Diese wird vom Benutzer
mit Hilfe der App erstellt und editiert. In den nächsten Abschnitte werden ausgehend von der Definition einer Konfiguration der allgemeine Aufbau und Mechanismen im Umgang
mit dieser eingeführt. Anschließend werden mögliche Datenformate und die Vorhersage von künftigen Konfigurationen diskutiert. Da es keine bestehenden Systeme gibt, welche 
Konfigurationen für die betrachteten Szenarien bereitstellen, wird ein Versuch unternommen eine eigene Spezifiaktion zu beschreiben.


\section{Definition einer Konfiguaration}
Im Kontext der Informatik ist der Begriff Konfiguration aus eher zwei unterschiedlichen Gebieten bekannt, so beschreibt dieser Term zunächst die Möglichkeit Parameter von
Programmen teilweise frei und in gewissen Grenzen anzupassen. So kennt jeder die Möglichkeit sein Mailprogramm zu konfigurieren und etwa neue Accounts hinzuzufügen. 
Ein weiteres mal tritt die Konfiguration in der theoretischen Informatik auf, so lässt sich das Konstrukt der Automtaten mit Hilfe von Konfigurationen beschreiben. Genauer
gesagt, wird der Zustandsübergang damit beschrieben\footnote{Des Weiteren sind z.B. in der Erzähltheorie und Chemie Konstrukte wie der Konfiguration vorzufinden. }. 
\\\\
In beiden Beispielen kommt zur Geltung, dass mit Hilfe einer Konfiguration der Zustand eines gewissen Konstruktes ausgehend vom Ersteller / Nutzer beschrieben wird. Diese
Beschreibung trifft ebenfalls auf InstantAmbient zu jedoch mit ein paar speziellen Anmerkungen. Zunächst können je nach Möglichkeit der Umgebung eine Vielzahl an 
Einstellungen vorgenommen werden und es wird sich nicht ausschließlich auf eine einzelne Konfiguration konzentriert, sondern je nach Situation eine neue angelegt oder editiert. Am 
besten lässt sich dies mit einer Vielzahl von Konfigurationen beschreiben die für eine spezielle Plattform angepasst sind. Mit Hilfe dieser Beschreibung lässt sich eine Definition für Konfiguration innerhalb von InstantAmbient beschreiben.

\newtheorem{mydef}{Konfiguration}
\begin{mydef}
Eine Konfiguration beschreibt den gewünschten Zustand einer Umgebung, welche von einem User definiert wird. Dabei beschreibt genau eine Konfiguration eine Umgebung. Als
Umgebung können beispielsweise Gebäude als ganzes, einzelne Räume oder gar Autos angesehen werden.
\end{mydef}

Diese Definition beschreibt die Konfiguration als solches sehr generell im folgenden wird diese ergänzt unter anderem einer Beschreibung wie diese aufgebaut ist. 

\section{Aufbau}
Nachdem allgemein Formuliert wurde, wie eine Konfiguration beschrieben ist, wird erläutert wie sich der Aufbau gestaltet. Hierfür werden mehrere Mechanismen erklärt, welche notwendig für die Erstellung einer Konfiguration sind. Diese beschreiben den allgemeinen Aufbau und gehen bis hin zu Konzepten wie der Polymorphie. 

\subsection{Allgemeine Strukturierung}
Zunächst soll geklärt werden wie sich der Aufbau einer Konfiguration gestaltet. Dazu sollten Überlegungen angestellt werden, welche Art von Daten und letztendlich Konfigurationsvarianten gespeichert werden müssen. Hierfür muss die Prämisse mit einbezogen werden, dass es sich um eine generische Gestaltung handeln soll, d.h. zu den Ansprüchen von Autos und etwa Hotelzimmern genügt. 
Ganz offensichtlich gibt es Unterscheidungen zwischen Attributen, einer Strukturierung dieser und eventuellen speziellen Mechanismen im Umgang mit diesen.
\\\\
Attribute, sind Zahlenwerte oder andere Datenstrukturen die repräsentiert werden. Dies sind die kleinsten Einheiten einer Konfiguration.
\\\\
Des Weiteren ist die Strukturierung und Organisation der Daten elementar für die Weiterverbreitung, so sollte das Format für Client, sowie den weiteren Bearbeitungsinstanzen leicht zu erstellen und parsbar sein. Welche Schwierigkeiten hier entstehen können, wird in den nächsten Abschnitten geklärt. 
\\\\
Im folgenden werden all die angesprochenen Strukturierungen in den einzelnen Abschnitten geklärt. Als Strategie wird die des Bottom-Up verfolgt, wobei mit den elementarsten Bestandteilen angefangen wird. 

\subsection{Attribute und ihre Typen}
Wie bereits angesprochen sind Attribute die kleinsten Einheit der Repräsentation von Konfigurationswerten. Dies ist notwendig, da beispielsweise Temperaturen andere Werte annehmen können, als eine Sammlung von unterschiedlichsten Radiosendern für das Autoradio. Dies bietet auch einen guten Einstieg in die Unterscheidung der einzelnen Typen. Damit einzelne Typen exakt zugeordnet werden, gehören diese immer zu einem Attribut. Diese Überlegung ist logisch und repräsentiert die bekannten Meachnismen, welche wir beispielsweise aus der Programmierung kennen, so kann das gezeigte Beispiel folgendermaßen interpretiert werden\footnote{Die hier gezeigte Syntax soll lediglich als Pseudocode dienen und nicht als festes format dienen.}:
\lstset{language=bash}
\begin{lstlisting}[caption=Zuweisung eines Attributs als Pseudocode , captionpos=b]
   x = 2
\end{lstlisting}
	
\\\\

So wäre im Falle einer Konfiguration x ein Attribut mit dem Wert 2, wobei es sich offensichtlich hier um eine Ziffer und letztendlich eine Zahl handelt. Zunächst müssen Attribute eindeutig sein und können einmalig deklariert werden. 
\\\\
Spannender ist die Frage welche Typen notwendig für die  Speicherung und Verarbeitung einer Konfiguration notwendig sind. Besonders Informatikern sind die Probleme bekannt, dass Datenstrukturen schnell komplex und unüberschaubar sind. Diesen Phänomen sollte vorgebeugt werden, sodass einfache Strukturen vorherrschen müssen. 
\\\\
Die Überlegung zwischen Raumtemperaturen und Fernsehsendern geben erste Hinweise auf notwendige Elementartypen. So sollte eine Temperatur als eine Fließkommazahl und ein Fernsehsender mit Hilfe dessen Namens, sprich einem Text definiert werden. Zahlen sollten ebenfalls als ganze definiert werden können. So sind schon komplexere Beispiele vorstellbar:
\lstset{language=bash}
\begin{lstlisting}[caption=Zuweisung mehrerer Attribute, captionpos=b]
  temperatur = 21.7
	sender_1 = "ARD"
	sender_2 = "ZDF"
\end{lstlisting}


Das letzte Listing zeigt bereits eine neue Klasse von Problemen, so ist die Sammlung einzelner Attribute, welche inhaltlich zum selben Kontext kompliziert. Die Einführung von Namenskonventionen zur Sammlung von Attributen, welche zu einer Domäne gehören, wären eine mögliche Lösung. Bei der Möglichkeit von wohl hunderten Radio- und Fernsehsendern, wird schnell klar, dass eine Unübersichtlichkeit garantiert ist. 	Als Konsequenz sollten Elementartypen in der Lage sein gesammelt zu werden. Im Endeffekt beschreibt dieser Mechanismus eine Liste, so ist:

\lstset{language=bash}
\begin{lstlisting}[caption=Sammlung mehrerer Attribute in einer Liste, captionpos=b]
   sender = ["ARD", "ZDF", "OpenTV", "OneMoreChannel"]
\end{lstlisting}

Im Vergleich zu: 

\lstset{language=bash}
\begin{lstlisting}[caption=Sammlung mehrerer Werte ohne Liste, captionpos=b]
  sender_1 = "ARD"
	sender_2 = "ZDF"
	sender_3 = "OpenTV"
	sender_4 = "OneMoreChannel"
\end{lstlisting}


Eine wesentliche Vereinfachung. Für die Zuweisung von Werten, sind keine weiteren Typen notwendig. 
\\\\
Die hier gewonnenen Erkenntnisse sollen wie zuvor im letzen Abschnitt formell in einer Definition festgehalten werden.

\newtheorem{mydef}{Attribut}
\begin{mydef}
Ein Attribut definiert eine Teileigenschaft einer Konfiguration. Dieses Attribut kann eine Zahl, einen Text oder eine Sammlung aus diesen beinhalten.
\end{mydef}
\\\\
Der nächste Abschnitt wird sich mit einem übergeordneten Konzept, der Sammlung von Attributen sammeln.


\subsection{Sektionen}
Mit der Einführung von Attributen und deren Typen ist es grundlegend möglich zunächst die notwendigen Werte einer Konfiguration zu speichern. Jedoch wird man bei diesem Konzept mit einer weiteren Klasse von Problemen konfrontiert, die im folgenden Listing gezeigt werden sollen:
\lstset{language=bash}
\begin{lstlisting}[caption=Attributssammlung ohne Sektionen, captionpos=b]
  temperatur_wohnzimmer = 21
	temperatur_badezimmer = 23
	radiosender_badezimmer = ["Fritz", "Radio1", "InfoRadio"]
	tvsender_wohnzimmer = ["BBC", "Phoenix"]
\end{lstlisting}

Das Beispiel zeigt zwei Dinge, dass Konfigurationen ausserordentlich komplex werden können und einzelne Teile einer Konfiguration offensichtlich einen bestimmten Kontext zugeordnet werden können. 
\\\\
Eine Kategorisierung bringt den Vorteil, dass der Inhalt einer Konfiguration noch mals unterteilt werden kann. Dies schafft nicht nur Übersichtlichkeit, sondern auch Klassifikationen. Besonders ist Zweiteres interessant für die spätere Weiterbearbeitung. So können bestimmte Systeme für eine bestimmte Klasse zuständig sein. Dieses Thema wird jedoch erst im nächsten Kapitel diskutiert. 
\\\\
Die Arten der Klassifikation können sehr unterschiedlich ausfallen. Die Einteilung ist stark von der verwendeten Umgebung und deren Ansprüchen abhängig. Bei einem Hotelzimmer wäre eine mögliche Einteilung folgende: 

\lstset{language=bash}
\begin{lstlisting}[caption=Mögliche Sektionen innerhalb einer Hotelkonfiguratione, captionpos=b]
  Wohnzimmer: 
		temperatur = 21
		tv_sender = ["ARD", "ZDF"]
		helligkeit = 0.7
	Badezimmer: 
		wasser_temperatur = 36
		radio_sender = ["Fritz", "Jazz Radio"]
		lautstaerke = 0.25
\end{lstlisting}

Wobei ein Auto folgende belange haben könnte:

\lstset{language=bash}
\begin{lstlisting}[caption=Mögliche Sektionen innerhalb einer KFZ-Konfiguration, captionpos=b]
   	Fahrersitz:
		hoehe = 12
		winkel = 96
		temperatur = 32
	Reifen: 
		vorne = 3.58
		hinten = 4
\end{lstlisting}


Die Beispiele zeigen, dass das gezeigte Hotelzimmer sich nach den Räumen richtet, wobei im Auto der Innenraum und sicherheitsrelevante Systeme. Es kann verdeutlicht werden, in wie weit die Unterscheidung und Sammlung einen Vorteil bringen. Attribute sind kontextuell voneinander getrennt ohne eine zu starke Hierarchie einzugehen, welche eventuell später schwerer zu bearbeiten wäre.
\\\\
Des Weiteren ergeben sich durch die Staffelung weitere interessante Konzepte, besonders in Hinsicht auf sicherheitsrelevante Systeme. Diese sollten bei der Verarbeitung und dem Laden der Daten eine besonders hohe Priorität genießen. Salopp gesagt, nützt es einen Fahrer nichts bereits seine Lieblingsmusik zu hören, wenn die Aussenspiegel noch nicht auf der korrekten Position sind. Die Einführung einer Priorisierung ist auf zwei unterschiedlichen Ebenen vorstellbar, auf ebene der Konfiguration und innerhalb einer Sektion: 
\lstset{language=bash}
\begin{lstlisting}[caption=Priorisierung von Sektionen, captionpos=b]
   	Fahrersitz:
		prioritaet = "hoch"
		hoehe = 12
		winkel = 96
		temperatur = 32
\end{lstlisting}


Die zweite Möglichkeit ist keine explizite Nennung der Priorisierung innerhalb der Konfiguration, sondern diese automatisch den bearbeitenden Systemen zu überlassen. 
Welche Stufen der Priorisierung möglich sein sollten, sind ebenfalls abhängig von den Systemen, zumindest sollte zwischen "`hoch"'' und "`niedrig"', wobei ersteres explizit genannt werden sollte. 
\\\\
Zusammenfassend wird das Konstrukt der Sektionen ebenfalls abschließend definiert.
\newtheorem{mydef}{Sektionen}
\begin{mydef}
Attribute und deren Typen lassen sich kontextuell in einzelne Sektionen gruppieren. Diese Gruppierungen können je nach Einsatzszenario unterschiedlich priorisiert werden. Die Nutzung von Sektionen ist nicht zwingend.
\end{mydef}
\\\\
Der nächste Abschnitt widmet sich dem mehrfachen Vorkommen von Attributen.

\subsection{Polymorphie}
Sektionen bieten neben der Klassifizierung von Attributen einen weiteren Vorteil, im ursprünglichen Konzept mussten unterschiedliche Temperaturen für einzelne Zimmer folgendermaßen beschrieben werden:

\lstset{language=bash}
\begin{lstlisting}[caption=Sammlung von Attributen ohne Polymorphie, captionpos=b]
  wohnzimmer_temperatur = 21
	badezimmer_temperatur = 22
\end{lstlisting}

Wenn sich streng an die aufgestellten Regeln der Sektionen und Attribute gehalten wird, wäre eine Beschreibung in diesem Stile möglich: 


\lstset{language=bash}
\begin{lstlisting}[caption=Beispiel einer Konfiguration ohne Polymorphie, captionpos=b]
  temperatur_allgemein = 20
	
	Wohnzimmer: 
		temperatur_wohnzimmer = 21
	
	Badezimmer: 
		temperatur_badezimmer = 22

\end{lstlisting}
 
Die Betrachtung der Konfiguration zeigt, dass ein erheblicher Overhead entsteht, alleine bei der Beschreibung der Temperatur. Es gilt global und für alle Zimmer die Eigenschaft der Temperatur, wenn Sektionen nicht genutzt werden, müssen diese unterschieden werden. Jedoch impliziert eine Sektion eine Zugehörigkeit eines Attributs zu einem bestimmten Kontext. Die eingeführte Hierarchie kann eine elementare Erleichterung bringen. Dieses Konzept ist ebenfalls als Polymorphie in der Programmierung bekannt. Das Attribut nimmt je nach dem in welchem Kontext es sich befindet eine andere Form an. Mit Hilfe des Konstrukts, lässt sich das Beispiel vereinfachen: 

\lstset{language=bash}
\begin{lstlisting}[caption=Beispiel einer Polymporphie basierten Konfiguration, captionpos=b]
  temperatur = 20
	
	Wohnzimmer: 
		temperatur = 21
	
	Badezimmer: 
		temperatur = 22

\end{lstlisting}

Es ist einheitlich die Sprache von einer Temperatur, welche in gewissen Situationen anders sein kann. Eine weitere Vereinfachung der Konfiguration.

\newtheorem{mydef}{Polymorphie}
\begin{mydef}
Mit Hilfe der Polymorphie lassen sich gleiche Attribute in einem unterschiedlichen Kontext beschreiben. Beispielsweise können Temperaturen in unterschiedlichen Räumen abgebildet werden.
\end{mydef}

\subsection{Ableitungen von Konfigurationen}
Die letzte Stufe bei der Beschreibung des Konfigurationsaufbau, ist die Bildung von Ableitungen. Der Gedanke hinter diesem Konstrukt ist die Wiederverwendbarkeit einzelner Teile einer Konfiguration. Ausgehend davon könnte jeder User seine ganz allgemeinen Präferenzen in einer Konfiguration gespeichert haben. Diese Treffen auf nahezu alle Umgebungen zu, so beispielsweise auch auf die Innentemperatur eines Autos zu. So finden sich Redundanzen innerhalb unterschiedlichster Konfigurationen wieder:
\lstset{language=bash}
\begin{lstlisting}[caption=Globale Attribute einer Konfiguration, captionpos=b]
	temperatur = 21
	helligkeit = 0.3
	luftfeuchtigkeit = 0.6

\end{lstlisting}

Sind nur einige der Daten die oftmals generell den persönlichen Präferenzen zuzuschreiben sind. In dieser Konsequenz sollte ein weiterer Mechanismus Teil der Konfiguration sein um solche Stammdaten mit einzubeziehen. So kann eine spezielle Konfiguration für ein Auto als eine Erweiterung der Stammdaten angesehen werden:
\lstset{language=bash}
\begin{lstlisting}[caption=Ableitung der Globalkonfiguration, captionpos=b]
	-> globale_konfiguration
	
	innnenraum:
		luftfeuchtifkeit = 0.4

\end{lstlisting}

Alle anderen bereits beschriebenen Konventionen wie etwa die Polymorphie gelten weiterhin. Das Endprodukt der Konfiguration beschreibt ebenfalls nur eine Datei und gibt keine Verlinkung auf andere vor. Jedoch bietet dieses Konzept nochmals eine übergeordnete Ableitung über alle Konfigurationen hinweg bekannt. So ist nicht nur die Einteilung mit Hilfe von Sektionen, sondern auch Ableitungen möglich. 

\newtheorem{mydef}{Abgeleitete Konfiguration}
\begin{mydef}
Eine Konfiguration lässt sich von einer anderen Ableiten. Alle Konzepte bleiben bestehen, das Endprodukt, ist eine neue Konfiguration innerhalb einer Datei. 
\end{mydef}

\section{Datenformate}
Die in diesen Kapitel verwendete Syntax zur Beschreibung der Konfiguration wurde zu Zwecken der Verständlichkeit eingeführt. Des Weiteren wäre eine vorzeitige Entscheidung für das verwendete Datenformat eine Einschränkung in der Konzeption des Formats. In diesem Abschnitt werden unterschiedliche Repräsentationen einer Konfiguration in XML und JSON vorgestellt. Diese sind die gängigsten Formate im Austausch von Daten. Im folgenden werden die einzelnen Darstellungen exemplarisch gezeigt und diskutiert. Ziel ist es für InstantAmbient eine passende Form der Datenrepräsentation zu finden. 

\subsection{XML}
XML ist wohl das bekannteste Format im Austausch von Daten es wird vielseitig eingesetzt und eine große Anzahl von Technologien basieren darauf. Die Tools und das Verständnis ist ausgereift und XML selbst ist für den generischen Ansatz von Konfigurationsbeschreibungen prädestiniert. Eine Beschreibung einer Beispielkonfiguration würde folgendermaßen aussehen: 

\lstset{language=XML}
\begin{lstlisting}[caption=XML-Konfiguration, captionpos=b]
	<configuration>
		<!-- Deklaration einer Zahl -->
		<number name="temperatur">21</number>
		
		<text name="willkommens_nachricht">Herzlich Willkommen</text>
		<!-- Eine Sammlung an Attributen -->
		<collection name="tv_sender" >
			<text>ARD</text>
			<text>ZDF</text>
			<text>Phoenix</text>
		</attribute>

		<!-- Sektionen -->
		<bathroom>
			<number name="temperatur">22</number>
		</bathroom>
	</configuration>
\end{lstlisting}

Das Beispiel soll ein Gefühl dafür erzeugen, wie eine Konfiguration im XML-Format aussehen könnte. Alle nötigen Vorgaben sind eingehalten worden. Attribute werden als solche abgebildet wobei der Name der Node den Typen oder eine Sektion beschreibt. Die Beschreibung ist relativ intuitiv und schnell zu erfassen. Die Hierarchien sind flach und schnell zu parsen. Die Nutzung von XML würde nur wenige Tücken mit sich bringen, so muss der Typ einzeln als Node-Name spezifiziert werden und das Attribut findet sich innerhalb des "`name"'. Das ist wohl die größte Schwäche. 

\subsection{JSON}
Als zweiter möglicher Formatkandidat, wurde sich für JSON entschieden. Dieses wird besonders im Web zur Serialisierung von Daten eingesetzt und zeichnet sich durch eine einfache Lesbarkeit und bereits spezifizierten Datentypen aus. Ein Beispiel zeigt die Eigenschaften von JSON:

\lstset{language=bash}
\begin{lstlisting}[caption=JSON-Konfiguration, captionpos=b]
{
	"temperatur" : 21,
	"willkommens_nachricht" : "Herzlich Wollkomen", 
	"tv_sender" : ["ARD", "ZDF", "Phoenix"], 

	"bathoroom" : {
		"tempertur" : 22
	}
}
\end{lstlisting}

Es wurde exakt das gleiche Beispiel wie zuvor im Abschnitt zu XML gewählt, es ist ersichtlich, dass JSON insgesamt schmaler und einfacher zu lesen ist. Des Weiteren sind die einzelnen Datentypen erkennbar.

\subsection{Formatentscheidung}
Eine Entscheidung für ein Format bringt immer Tücken mit sich und es müssen vorsichtig Vor- und Nachteile abgewägt werden. Wünschenswert wäre eine Multiformat-Unterstützung da im Endeffekt XML und JSON die gleichen Belange im Grunde nur unterschiedlich codieren. Im Falle von InstantAmbient wird sich für JSON entschieden. Es gibt einige entscheidende Vorteile neben den fest integrierten Datentypen, sind die Dateien kompakter und es gibt eine Reihe an performanten Parsern. Insgesamt tritt weniger Overhead auf wie bei klassichen Markup-Sprachen. Diese Entscheidung soll jedoch kein Dogma sein, so wird die Archtiektur es vorsehen, dass diese Formate änderbar sind. 

\section{Weitere Konzepte}
Als letzter Teil dieses Kapitels, sollen mögliche Erweiterungen und Konzepte von Konfigurationsdateien beschrieben werden. Diese Mechanismen hängen nicht nur von dem Format selbst, sondern auch anderen Systemen innerhalb der Konfigurationsbearbeitung ab. 

\subsection{Echtzeit-Änderungen}
Im Laufe dieses Kapitels wurden Konfigurationen als ein reines statisches Konstrukt betrachtet. Zwar wird eine gewisse Dynamik durch die Einführung von Sektionen, Polymorphie und Ableitungen impliziert, dennoch muss die Konfiguration immer als ganzes Übertragen werden. Diese beschreibt so vollständig wie möglich die Wünsche der Nutzers für eine komplette Umgebung. Haben sich lediglich einzelne Teile geändert, müssen diese dennoch komplett übertragen werden. Dies erzeugt Redundanzen und unnötigen Traffic innerhalb der Systeme. Es ist sich leicht vorzustellen wie schnell die Komponenten verwirrt sein könnten. Des Weiteren ist beispielsweise die Raumtemperatur kein Manifest, dass während der Gesamtnutzung der Umgebung unveränderbar ist. Bedürfnisse ändern sich, sei es die Innentemperatur oder der Radiosender, welcher gerade ausgewählt wurde. Einzelne Eigenschaften sollen und müssen in Echtzeit geändert werden. Die bedeutet, dass als Erweiterung jederzeit partielle Ausschnitte der Konfiguration geändert und übertragen werden können ohne das Neuladen der kompletten Konfiguration und dessen Zusammenhang der Umgebung zu provozieren.

\subsection{Vorhersagen}
Eine zweite mögliche Erweiterung ist die Vorhersage von möglichen Konfigurationen, ähnlich der Produktempfehlungen bei großen Onlinekaufhäusern. Umgebungen sind dadurch definiert, dass sie sich Eigenschaften teilen auch wenn sie unterschiedlich sind, jedoch ändern sich diese auch nach Art der Umgebung. So sind Sitzpositionen innerhalb von Autos abhängig vom Fahrzeugtyp dem eigentlichen Sitz et cetera. Durch selbstlernende Algorithmen und eine entsprechend große Datenbasis wäre es möglich den Nutzern Konfigurationsempfehlungen auszusprechen auf Basis der Kenntnisse über das eigene und fremde Profile. Vorhersagen ersparen aufwändige Neukonfigurationen und können besonders Aspekte auf Sicherheit und andere wichtige Themen legen. Eines der Kernprobleme in diesem Feld ist, dass die Algorithmen sich auf eine relativ große Datenbasis verlassen. Hier entstehen Fragen darüber wie sich die Datenbasis generiert, wenn jeder Nutzer die Konfigurationen auf seinen Smartphone speichert. Gibt es in diesem Falle nur den Weg über die Cloud oder könnte die Umgebung selbst als Datenvermittler und Wissensspeichen. Ausserdem stellen sich Fragen zum Datenschutz und der Datenintegrität. 

\subsection{Smart-Environments}
Ausgehend von den Vorhersagen zu neuen Profilen und der Intention, dass Systeme in der Lage sind zu lernen, welche Einstellungen Nutzer treffen, kann dieser Effekt auch auf der anderen Seite genutzt werden. Das Stichwort hier, sind Smart-Environments, wenn sich bestimmte Muster finden lassen, beispielsweise wann das Licht ausgeschaltet wird. Ressourcen können intelligent geplant und eingeteilt werden. Die Möglichkeiten sind vielfältig und können besonders zur Energie- und Kosteneinsparungen genutzt werden.

\subsection{Einführung neuer Typen}
Die letzte mögliche Erwägung ist die Einführung eines neuen Datentyps. Wie gezeigt können Konfigurationen schon jetzt dynamisch und skalierend angelegt werden. Die Unterscheidung zwischen Number, String und Collections sind hierfür elementar. Die Praxis wird zeigen ob ein Boolean-Typ von Nöten ist, welcher in der Lage ist, einfache Zustände darzustellen.
\lstset{language=bash}
\begin{lstlisting}[caption=Möglicher Boolean-Typ, captionpos=b]
	active = true
\end{lstlisting}

In Zukunft könnten hiermit größere Erleichterungen möglich sein. Im allgemeinen kann nur der Einsatz und der Umgang mit Konfigurationen zeigen, welche Dinge notwendig sind und noch gebraucht werden. 




\chapter{Der Client}

Dieses Kapitel beschäftigt sich mit dem Android Client der den Projektnamen AmbientClient trägt. Hierbei wird auf den Aufbau des Client sowie 
den Ablauf einer anzulegenden Umgebung eingegangen.

\section{Anforderungen}
Der Client muss im Stande sein mitzubekommen, dass er sich mit einer neuen Umgebung befindet. Des Weiteren muss er bei einer schon vorhanden Konfiguration 
für die Umgebung im Stande sein, eine Veränderung der Konfigurationen automatisch zu senden ohne das der Benutzer etwas machen muss.
Er muss aufzeigen welche Möglichkeiten die Umgebung bietet. Des Weiteren soll er dem User eine Übersich für seine angelegten Umgebungen bieten und durch kurze Wege 
eine leichte und schlüssige Benutzerführung ermöglichen.   

\section{Aufbau} 
Wie schon im vorherigen Kapitel beschrieben soll der Client die Daten des Benutzer auf seinem Smartphone speichern können. 
Wenn dieser sich dann in seiner Umgebung einfindet sollen die Daten übertragen werden. 
Dabei gibt es natürlich verschieden Aspekte die man betrachten muss. Zum einen, wann der Client die Daten übertragen soll. 
Ansonsten gibt es noch die Frage wann der Client eine neue Umgebung anlegen soll und ob dieses Automatisch passieren soll oder ob der User dies selber machen soll.
\\\\
Es ist natürlich am Benutzerfreundlichsten für den User wenn er so wenig wie möglich selber machen muss, 
ohne das die Anwendung in bevormundet und einfach Einstellungen vornimmt, ohne das der User dies möchte. 
Daher ist es so, dass man beim initialen Start der Anwendung seine Allgemeine Konfiguration eingeben muss. 
Der AmbientClient soll automatisch erkennen wenn er sich in einer neuen Umgebung befindet. 
\\
Nun kann der User für seine Umgebung die beim erstmaligen anlegen für das Hotel oder Auto allgemein gültig ist eine Konfiguration anlegen. 
Besucht der Benutzer zu einem späteren Zeitpunkt das gleiche Hotel, aber bekommt ein besseres Zimmer mit mehr Möglichkeiten. 
So wird beim betreten des Raumes und öffnen der AmbientClients
die Konfiguration übertragen und zugleich er hält der Client die Information das mehr Möglichkeiten zum Konfigurieren in diesem Zimmer vorhanden sind. 
Wenn dies der Fall ist bietet der AmbientClient dem User an eine weitere Umgebung für das Zimmer anzulegen und die zusätzlichen Konfigurationen vorzunehmen.




\chapter{Das Backend}

\section{Anforderungen}
\section{InstantConnector}
\section{InstantBrain}
\section{InstantActor}

\chapter{Ausblick \& Zusammenfassung}
Abschliessend soll dieses Kapitel eine Zusammenfassung über die Erkenntnisse und zukünftigen Möglichkeiten des Forschungsprojektes InstantAmbient geben. 
Nachdem die Anforderungen des Projektes formuliert wurden, konnte ein Überblick zu bestehenden Lösungen gegeben werden. Als ersten großes Thema kam es zur Definition von Konfiguration mit der Erklärung ihrer speziellen Mechanismen und Funktionen. Die Grundlage der Konfigurationen erlaubten es spezielle Lösungsmöglichkeiten und eine Architektur für den Proof of Concept zu entwickeln. Die darauf folgenden Kapitel beschäftigten sich mit dessen Umsetzung.

\section{Gewonnene Erkenntnisse dieser Arbeit}
Insgesamt kann ein positives Resümee über dieses Projekt gezogen werden. Es wurde gezeigt, dass die an InstanAmbient gestellten Anforderung durchaus umsetzbar sind. Mit der Definition von Konfigurationen konnte ein Grundstein für auf diesen Konzept basierende Systeme gelegt werden. Die gewonnenen Erkenntnisse über die Architektur und dem Einsatz der einzelnen Systeme hat gezeigt wie vielfältig das Projekt einsetzbar ist. Die Entscheidung über das Client orientierten Konzept und somit einen klaren Schnitt zwischen Nutzerinteraktion und verarbeitenden Systemen hat sich als richtig erwiesen. Das Systemdesign ist klar strukturiert und in jedem Falle erweiterbar sei es beim Front- oder Backend. Es konnte ein Grundstein für den Ausbau der Systeme gelegt werden.

\section{Probleme während der Projektphase}
Probleme während des Projektes gab es auf mehreren Ebenen, da die Fällung einzelner Entscheidungen Auswirkungen auf das gesamte System haben. Besonders die Entscheidung zwischen dem Client- oder Backend orientierten Ansätzen hatten Auswirkungen auf das gesamte Konzept. Die getroffene Wahl war die richtige. Besonderes Kopfzerbrechen brachte die Gestaltung der UI der Android-App und das Konfigurationsrouting mit sich. Auf jeder Ebene hätten schlechte Lösungen das Projekt zum Fall bringen können. 

\section{Zukünftige Möglichkeiten & Herausforderungen}
Um einen Blick in die Zukunft zu werfen sind mehrere Weiterentwicklungen ausgehend von den hier gewonnenen Erkenntnissen möglich. 
\\\\
Der wohl nächste größere Schritt wäre eine größere Testreihe des Systems in heterogenen Umgebungen. So wären sicherlich unterschiedlichste Szenerien wie etwa die der Fahrzeuge oder Hotelumgebungen denkbar. Als erster Schritt wäre zunächst die Portierung der Beispielimplementierung auf die einzelnen Zielsysteme notwendig. Des Weiteren müsste eine Koppelung mit bestehenden Bus-Systemen etc. stattfinden. Dies wird wohl die herausforderndste Aufgabe sein und zeigen ob die Konzeptionen der Connectors und Actors den Anforderungen dieser Systeme entspricht. Eine Vielzahl von Umgebungen erlauben es weiterhin nicht ohne weiteres in die Infrastruktur einzugreifen. Im Bereich der Automobile ist dieses Phänomen besonders stark. Hierfür sind mit größer Wahrscheinlichkeit Kooperationen mit Herstellern notwendig. Der größte wünschenswerte Fall wäre natürlich der Produktiveinsatz. 
\\\\
Eine weiterer Forschungsaspekt wäre die Untersuchung möglicher Konfigurationsvorhersagen basierend auf dem Wissen um welche Umgebung es sich handelt und einer entsprechenden Datenbasis. Hierbei gilt es die Fragen zu klären woher diese Daten stammen und auf welcher Basis eine Analyse erfolgen kann. In InstantAmbient stecken eine Menge weiterer Möglichkeiten und wir selbst sind gespannt was daraus wird. 
\include{chap10/chap10}

%Anhang
%\appendix

\newpage
% Glossar
%\printglossaries
%\addcontentsline{toc}{chapter}{Glossar}
% Abbildungsverzeichnis
%\listoffigures
% Abbildungsverzeichnis
%\chapter{Verzeichnisse}
%\listoffigures
%\addcontentsline{toc}{chapter}{Abbildungsverzeichnis}
%\chapter{Abbildungsverzeichnis}

%Code-Verzeichnis
%\renewcommand{\lstlistlistingname}{Verzeichnis der Quelltextauszüge}
%\lstlistoflistings
%\addcontentsline{toc}{chapter}{Verzeichnis der Quelltextauszüge}

% Bibtex mit der externen Datei bibliography.bib
%\nocite{*}
%\bibliography{bibliography}
%\bibliographystyle{alpha}
%\addcontentsline{toc}{section}{Literaturverzeichnis}

% Index
%\clearpage
%\addcontentsline{toc}{chapter}{Index}
%\printindex
\end{document}
